\documentclass[11pt,a4paper]{article}
\usepackage[utf8]{vietnam}
\usepackage{import}
\usepackage{caption}
\usepackage{fancyhdr}
\usepackage{hyphenat}
\usepackage{amsmath}
\usepackage{amsfonts}
\usepackage{amssymb}
\usepackage{graphicx}
\usepackage{makecell}
\usepackage{multirow}
\usepackage[left=2cm,right=2cm,top=2cm,bottom=2cm]{geometry}


\renewcommand\theadalign{bl}
%\renewcommand\theadfont{\bfseries}
%\renewcommand\theadgape{\Gape[4pt]}
%\renewcommand\cellgape{\Gape[4pt]}

\begin{document}
\begin{center}
\begin{tabular}{|p{2cm}|p{2cm}|p{12cm}|}
%\hline
%\thead{A Head} & \thead{A Second \\ Head} & \thead{A Third \\ Head} \\
\hline
\thead{Chương} & \thead{Phần} & \thead{Nội dung}\\
\hline
\multirow{2}{*}{Giới thiệu} & Giới thiệu bài toán & Mô tả bài toán: phát hiện và transform giấy tờ tùy thân xuất hiện trong ảnh\\
\cline{2-3} & Hướng tiếp cận truyền thống & Phương pháp đã được triển khai: phát hiện 4 góc của chứng minh thư như là 4 đối tượng riêng biệt rồi tổng hợp kết quả. \newline
Hạn chế:
\begin{itemize}
\item Tạo dữ liệu training: bounding box của góc, kích thước, diện tích, vị trí, ... 
\item Không tận dụng được mối liên quan giữa các góc của chứng minh thư trong ảnh 
\item Các góc có các đặc tính tương đối giống nhau, xảy ra trường hợp mô hình phát hiện ra các góc trùng nhau
\item Chỉ áp dụng được cho trường hợp có nhiều nhất 1 chứng minh thư trong ảnh 
\end{itemize}
\\
\hline 
\multirow{3}{*}{Cơ sở lý thuyết} & Học máy và học sâu & \\
\cline{2-3} & Mạng openpose & abc \\
\cline{2-3} & abc & cdba \\
\hline


\end{tabular}
\end{center}
\end{document}